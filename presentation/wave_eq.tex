\documentclass{beamer}
% Option [handout] in der Beamer-Klasse fuehrt dazu, dass alles gleich 
% aufgedeckt ist. Nuetzlich, falls die Folien verschickt oder gedruckt 
% werden sollten

%----TITEL----------------------------------------------------------%
\author[Barthel, Funke, Richter]{R. Barthel, G. Funke, N. Richter}
\title[Finite Differenzen Methode f\"ur die Schr\"odinger Gleichung]{Finite Differenzen Methode f\"ur\\ die Schr\"odinger Gleichung}
\institute[Numerik-Praktikum 24/25]{Numerisches Praktikum\\
                    WS 2024/2025\\
                      Universit\"at Leipzig}
\date[WS 2025]{3.\ Februar 2025}


%-------------------------------------------------------------------%


%-----OPTIONEN BEAMER-----------------------------------------------%
\usetheme{default}

\setbeamertemplate{headline}{}
\setbeamertemplate{footline}{%
         \phantom{x}\hspace{5pt} \insertshortauthor  
         \qquad\qquad \insertshorttitle 
         \qquad\qquad\qquad \insertshortdate 
         \hfill S.\,\insertframenumber\phantom{xy}\vskip3pt}
\usefonttheme[onlymath]{serif}
%-------------------------------------------------------------------%

\usepackage{amsmath,amsthm,amsfonts,amssymb,graphicx,animate}

\usepackage[ngerman]{babel}
%\usepackage[utf8x]{inputenc}  % in alten Latex-Versionen fuer die Eingabe z.B. 
                               % von Umlauten

\usepackage{pgfplots}
\pgfplotsset{compat=1.18}
\usepackage{tikz}
\usetikzlibrary{arrows,shapes,positioning}

\usepackage{listings}
\lstset{basicstyle=\scriptsize\ttfamily,frame=lines,breaklines=true}





%----------------------- Macros and Definitions --------------------------

 \definecolor{mycyan}{rgb}{0.0, 1.0, 1.0}

% hier koennen Befehle definiert werden
\newcommand{\Div}{\operatorname{div}}


\begin{document}



%%%%%%%% Titlepage%%%%%%%%%%%%%%%%%%%%%%%%%%%%%%%%%%
\frame{\titlepage}
%%%%%%% Introductory slides%%%%%%%%%%%%%%%%%%%%%%%%%

\begin{frame}
\frametitle{Inhalt}
\tableofcontents
\end{frame}

\section{Aufgabenstellung}

\begin{frame}
    \frametitle{Problemstellung}

    Die Schrödinger Gleichung auf $\mathbb{R}$ ist gegeben durch
    \begin{align*}
        iu_t - u'' + \lambda |u|^2u &= 0 \\
        u(x,0) &= \phi(x)
    \end{align*}
    
    mit $u:\mathbb{R}\times\mathbb{R}\to\mathbb{C}$, $\phi:\mathbb{R}\to\mathbb{C}$ und $\lambda\in\mathbb{R}$\\
    \ \\
    Im Folgenden gilt:
    \begin{itemize}
        \item $\lambda = -2$
        \item $u(x,t)=\frac{3}{2}\exp(i(\frac{5}{4}t-x))\text{sech}(\frac{3}{2}(x+5)-3t)$
        \item $\phi(x)=u(x,0)$
    \end{itemize}
\end{frame}

%Herleitung für Diskretisierung?

\begin{frame}
    \frametitle{Herleitung}

    Zeitzentrierte implizite Diskretisierung {\small(Crank-Nicolson)}:

    Sein $\Delta t, \Delta x>0$.
    Approximiere $\partial_t u = \partial_x^2 u - \lambda |u|^2u$ zur Zeit $\Delta t (k+\frac{1}{2})$ im Ort $\Delta x j$ durch
    \begin{align*}
        \frac{U^{k+1}_j-U^{k}_j}{\Delta t} &= \frac{1}{2}(\frac{U^{k}_{j+1}-2U^{k}_j+U^{k}_{j-1}}{\Delta x}+\frac{U^{k+1}_{j+1}-2U^{k+1}_j+U^{k+1}_{j-1}}{\Delta x})\\
        &\quad\qquad -\frac{\lambda}{2}(|U^{k+1}_j|^2U^{k+1}_j+|U^{k}_j|^2U^{k}_j)
    \end{align*}
    Dabei ist
    \begin{align*}
        \frac{\lambda}{2}(|U^{k+1}_j|^2U^{k+1}_j+|U^{k}_j|^2U^{k}_j) \approx \frac{\lambda}{4}(|U^{k+1}_j|^2+|U_j^k|^2)(U^{k+1}_j+U_j^k)
    \end{align*}
    

\end{frame}

\begin{frame}
    \frametitle{Herleitung}

    Zusammenfassen der Ortskomponenten und umschreiben mit 
    \begin{align*}
        A = \begin{bmatrix}
            2  & -1 &  0  &  \cdots  & 0 \\
            -1  & 2 &  -1  &  \cdots  & 0 \\
            0  & -1 &  2  &  \ddots  & 0 \\
            \vdots & \vdots & \ddots & \ddots & -1 \\
            0 & 0 & 0 & -1 & 2\\
        \end{bmatrix}
    \end{align*}
    liefert
    \begin{align*}
        i\frac{1}{\Delta t}(U^{k+1}-U^k)&+\frac{1}{2|\Delta x|^2}(AU^{k+1}+AU^k)+F_k(U^{k+1})=0\\
        (F_k(U))_j &= \frac{\lambda}{4}(|U_j|^2+|U_j^k|^2)(U_j+U_j^k)
    \end{align*}
    Dies kann umgeschrieben werden zu
    \begin{align*}
        U^{k+1} = (\frac{i}{\Delta t}I + \frac{A}{2\Delta x^2})^{-1}((\frac{i}{\Delta t}I - \frac{A}{2\Delta x^2})U^k - F_k(U^{k+1}))
    \end{align*}

\end{frame}

\begin{frame}
    \frametitle{Diskretisierung}

    Sein $\Delta t, \Delta x>0$ mit $J=\frac{30}{\Delta x},K=\frac{6}{\Delta t}\in\mathbb{N}$.\\
    Approximiere $u(j\Delta x, k\Delta t)$ durch $U_j^k$ f\"ur $j=-J, .. , J,\ k=0, .. , K$\\
    \ \\
    Startbedingung: $U_j^0=\phi(j\Delta x)\ \forall j\in\{-J, -J+1, .., J\}$\\
    %Randbedingung: $U_{-J}^k=U_{J}^k=0\ \forall k\in\{0, .., K\}$\\
    % Randbedingungen sind nicht gegeben
    Fixpunktitteration:
    \begin{align*}
        U_p = (\frac{i}{\Delta t}I + \frac{A}{2\Delta x^2})^{-1}((\frac{i}{\Delta t}I - \frac{A}{2\Delta x^2})U^k - F_k(U_p))
    \end{align*}
    mit $| U_{i+1}-U_{i} |<\epsilon$ als Abbruchskriterium und
    \begin{align*}
        (F_k(U))_j = \frac{\lambda}{4}(|U_j|^2+|U_j^k|^2)(U_j+U_j^k)
    \end{align*}
\end{frame}

\begin{frame}
    \frametitle{Eine Lösung}

    %Insert NLS_abs.gif here

\end{frame}

\section{Energieerhaltung}

\begin{frame}
    \frametitle{Energie}

    Erhaltungsgößen von Lösungen der PDG
    \begin{align*}
        \int_\mathbb{R} |u|^2 dx &= \text{const}\\
        \int_\mathbb{R} \frac{1}{2}|u'|^2 + \frac{\lambda}{4}|u|^4 dx &= \text{const}
    \end{align*}
    \ \\
    Diskretisierte Energien
    \begin{align*}
        \Delta x \sum_{j=-J}^J |U_j^k|^2 &= \Delta x \sum_{j=-J}^J |U_j^0|^2\\
        \frac{\Delta x}{2} \sum_{j=-J}^{J-1} |\frac{U_{j+1}^k-U_j^k}{\Delta x}|^2+\frac{\lambda}{4}\Delta x \sum_{j=-J}^J |U_j^k|^4 &= \text{const}
    \end{align*}

\end{frame}

\begin{frame}
    \frametitle{Energieverlust}

    \begin{figure}
        \begin{center}
            \includegraphics[width=0.7\textwidth]{eps_diss.png}
        \end{center}
    \end{figure}

\end{frame}


\section{Phasenproblem}

\begin{frame}
    \frametitle{Das Problem:}

    %NLS_real.gif und NLS_imag.gif

\end{frame}


\section{Explizite Lösung}

\begin{frame}
    \frametitle{<title>}

    

\end{frame}

%\begin{frame}
%    \frametitle{Gezeichnetes Bild}
%    
%    \begin{minipage}{0.65\textwidth}
%        Bei der Erkl\"arung kann ein Bild n\"utzlich sein.
%        Dieses hier wurde mit dem Paket tikz erstellt.
%    \end{minipage}
%    \hfill
%    \begin{minipage}{0.2\textwidth}
%        \begin{tikzpicture}[x=7.5mm,y=12.5mm]
%            \draw (0,0)--(1,1)--(2,0)--cycle;
%            \foreach \a/\b in {0/0,1/1,2/0}
%                { \fill (\a,\b) circle (2pt); }
%        \end{tikzpicture}
%    \end{minipage}
%\end{frame}




\section{Implementation}

%\begin{frame}[fragile]
%    \frametitle{Implementierung}
%    
%    Der Algorithmus ist wie folgt implementiert
%\begin{lstlisting}[numbers=none]   
%xneu = xalt - f(xalt)/df(xalt);
%k = k+1;
%\end{lstlisting}
%\end{frame}



\section{Numerische Ergebnisse}

%\begin{frame}
%    \frametitle{Numerische Ergebnisse -- Tabelle}
%    
%    \begin{center} %zentriert die Tabelle 
%    \begin{tabular}{l|cc|r} % erstellt eine Tabelle mit 4 Spalten, 
%                            % die erste ist linksbuendig, die zweite 
%                            % und dritte zentriert, die letzte
%                            % rechtsbuendig.
%                            % Nach der ersten und dritten Spalte ist ein 
%                            % vertikaler Strich
%        Parameter 1 & Ergebnis 1 & Ergebnis 2& Ergebnis 3 \\
%        \hline % fuegt horizontale Linie ein
%        2 & 0.5 & 0.4 & 0.3\\
%        3 & 0.5 & 0.4 & 0.1\\
%        5 & 0.2 & 0.4 & 0.3\\
%    \end{tabular}
%    \end{center}
%\end{frame}

%\begin{frame}
%    \frametitle{Numerische Ergebnisse -- externe Abbildung}
%    
%    \begin{figure}
%        \begin{center}
%            \includegraphics[width=0.7\textwidth]{Lerche-mit-koordinatensystem}
%        \end{center}
%    \end{figure}
%
%    
%    
%\end{frame}


%\begin{frame}
%    \frametitle{Numerische Ergebnisse -- Abbildung mit tikz}
%    
%    % Daten sind in Datei errors4tikz gespeichert
%    
%    \begin{figure}
%        \begin{center}
%            \begin{tikzpicture}[scale=0.9]
%                \begin{loglogaxis}[ymin=1e-5, ymax=1, xmin=1e0, xmax=1e5,
%                    legend pos=south west,legend cell align=left,legend style={fill=none}]
%                    \pgfplotsset{
%                    legend style={
%                    at={(-0.2,.5)}, anchor=east},
%                    font=\sffamily\scriptsize,
%                    xlabel=ndof,ylabel=error % add labels to x and y axis
%                    }
%                    %
%                    \addplot[solid,black,mark=square] table[x=ndofs,y=errors1] {errors4tikz.dat};
%                    \addlegendentry{ $\|\nabla(u-u_h)\|_{L^2(\Omega)}$}
%                    %
%                    \addplot[dashed,black, mark=triangle,mark options={solid}] table[x=ndofs,y=errors2] {errors4tikz.dat};
%                    \addlegendentry{ $\|\nabla(u-w_h)\|_{L^2(\Omega)}$}
%                    % add slope triangle
%                    \draw (axis cs:10,1e-4)--(axis cs:1000,1e-4)
%                                -- (axis cs:10,1e-2)
%                                -- cycle;
%                    \node at (axis cs: 100, 1e-4) [below] {$1$};
%                    \node at (axis cs: 10, 1e-3) [left] { $1$};
%                    %
%                \end{loglogaxis}
%            \end{tikzpicture}
%        \end{center}
%        \caption{\label{f:convhistory}Ein Konvergenzplot.}
%    \end{figure}
%\end{frame}



\section{Schlu{\ss}folgerungen}

%\begin{frame}
%    \frametitle{Schlu{\ss}folgerungen}
%    \begin{itemize}
%     \item Verfahren A ist effizienter als Verfahren B
%     \item Beide Verfahren liefern gute Ergebnisse
%    \end{itemize}
%\end{frame}


%\begin{frame}
%    \frametitle{Weiteres}
%    \begin{itemize}
%        \item Beachten Sie Warnungen und badboxes (z.b.
%            \emph{Overfull hbox} hei{\ss}t, dass der Text/das Bild/die Tabelle
%             zu breit ist, \emph{Overfull vbox}, dass es in vertikaler Richtung
%             zu gro{\ss} ist)
%        \item<2> Aufs Wesentliche reduzieren: Grober Anhaltspunkt zur Planung des
%            Vortrags: etwa 2 Minuten pro Folie
%    \end{itemize}
%
%\end{frame}




%\begin{frame}
%    \frametitle{Literatur}
%
%    \visible<1->{
%        \bigskip
%        {\scriptsize
%            \begin{thebibliography}{x}
%                \bibitem{hermann2011}
%                    M. Hermann.
%                    \newblock Numerische Mathematik.
%                    \newblock M\"unchen: Oldenbourg Verlag, 3.\ \"uberarb.\ und erw.\ Aufl., 2011.
%            \end{thebibliography}
%        }
%    }
%
%    \label{page:end}
%\end{frame}

\end{document}
